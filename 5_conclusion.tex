\chapter{Conclusion}   \label{chp_conclu}
In this thesis, we first expose that the bottleneck of legacy 802.11 is located at its MAC, instability of DCF and unfair queueing problem. 
Then we introduce 802.11ax as a solution which aimed at high efficient WLAN (HEW).
One of the important features of 802.11ax is OFDMA-based random access, which is MU random access in IEEE 802.11ax MAC. 
The system parameters are configured by AP dynamically, so it is more flexible and more complicated. 

Afterwards, we extend Bianchi's Markov chain model of DCF to model the OFDMA-based random access for its simplicity and accuracy. 
With the model we do saturation analysis to depict the steady state behavior of the mechanism.
Simulation validates the model. 
Then we could estimate the impact of a parameter set $\lbrace M, OCW_{min}, OCW_{max} \rbrace$ on system efficiency and access delay, where $M$ is the number of RUs for random access,  and $OCW_{min}, OCW_{max}$ is the initial and maximum OFDMA contention window. 
At last, rules of configuration of the parameter set aimed at approaching the optimal transmission probability $\tau^\star$ is proposed for AP with or without given number of contending stations.