\chapter{Conclusion}   \label{sec_conclu}
In this paper, we illustrate one of important features of the IEEE 802.11ax MAC, OFDMA-based random access. 
Different from legacy 802.11, the OFDMA-based random access mechanism is more complicated and flexible, not only multiple channels being allocated for random access, but also the parameter set being configured by AP in real time.
We generate a new Markov chain model of the OFDMA-based random access and validate our model by simulation. 

With the model, we derive the maximum system efficiency and minimum access delay, and estimate the effect of a parameter set $\lbrace M, m, OCW_{min} \rbrace$ on system efficiency and access delay, where $M$ is the number of RUs for random access, $m$ is the maximum backoff levels, and $OCW_{min}$ is the initial OFDMA contention window. 
All the three parameters have great impact on performance.
An interesting result is that the maximum system efficiency and minimum access delay are obtained by the same transmission probability $\tau$, which is given by $\tau^\star = min \lbrace 1,M/n\rbrace$. 
Then rules of configuration of parameter set aimed at reaching the optimal transmission probability $\tau^\star$ is proposed for AP according to system state, mainly the number of contending stations. 
The last group cases of various parameter sets validates our proposed rules.
