\chapter{Conclusion}   \label{chp_conclu}
In this thesis, we expose bottleneck of legacy 802.11---a SU random access for its instability property. 
Then we illustrate OFDMA-based random access which is MU random access in IEEE 802.11ax MAC. 
And the dynamic configuration of parameters makes it more flexible and more complicated. 

Afterwards, we extend Bianchi's Markov chain model of DCF to model the OFDMA-based random access for its simplicity and accuracy. We also validate our model by simulation. 
With the model, we could estimate the impact of a parameter set $\lbrace M, OCW_{min}, OCW_{max} \rbrace$ on system efficiency and access delay, where $M$ is the number of RUs for random access,  and $OCW_{min}, OCW_{max}$ is the initial and maximum OFDMA contention window. 
At last, rules of configuration of the parameter set aimed at approaching the optimal transmission probability $\tau^\star$ is proposed for AP with or without given number of contending stations.