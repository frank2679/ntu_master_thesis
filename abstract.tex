\begin{abstractzh}
近年來wifi成為分布最為廣泛的WLAN,wifi配置地越來越多,越來越密集,也有越來越多的設備接入wifi,通過wifi的流量也越來越大。
漸漸地,傳統802.11 接取層(MAC)基於分布式競爭接入(DCF)的方式本身的問題越來越凸顯出來,主要表現為碰撞加劇,從而網路吞吐量降低,接入時延變大,能耗嚴重。
我們稱之為密集分布場景(dense deployment scenario)的問題,所以IEEE 802.11成立ax工作組針對密集分布場景的問題,在傳統802.11 接取層(MAC)和物理層(PHY)都進行很大的修改。引入了OFDMA,實現多用戶信道(MU)。
802.11ax提出基於OFDMA的隨機接取,它比傳統的隨機接取更加覆雜,更有彈性。
本論文旨在分析該機制的效能,建立二維離散時間馬爾可夫模型很好地描述了其穩定狀態行為。
並且借此模型分析重要系統參數對系統效能的影響從而得出配置參數的方法。

\noindent
關鍵詞:802.11ax,多用戶信道,OFDMA,隨機接取
\end{abstractzh}

\begin{abstracten}
In recent years, WiFi has been the most extensively deployed WLAN. 
With more and more devices or users and exploding traffic, the WLAN is more and more dense. 
Gradually, problem from the nature of distributed coordination function (DCF), which is the foundation of legacy 802.11 MAC, has arisen and is becoming more and more severe. 
That is, severe collision causes degradation of throughput, defer access, and waste energy of stations, which in total is called dense deployment problem. 
Thus, task group IEEE 802.11ax, which was set up in 2014 confronting the above problem, makes revolutional modification of both MAC and PHY to legacy 802.11. 
OFDMA is issued in 802.11ax, implementing Multi-User (MU) channel. 
And correspondingly, 802.11ax proposes OFDMA-based random access, which is more complicated and flexible than legacy random access. 
This work model the OFDMA-based random access with bi-dimensional discrete-time Markov Chain to accurately depict its steady state behavior. 
With this model, we evaluate the effect of several system parameters and propose rules of configuring the parameters.



\noindent
\textbf{Keyword: 802.11ax, MU PHY, OFDMA, random access}
\end{abstracten}

\begin{comment}
\category{I2.10}{Computing Methodologies}{Artificial Intelligence --
Vision and Scene Understanding} \category{H5.3}{Information
Systems}{Information Interfaces and Presentation (HCI) -- Web-based
Interaction.}

\terms{Design, Human factors, Performance.}

\keywords{802.11ax, MU PHY, OFDMA, random access}
\end{comment}
