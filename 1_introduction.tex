\chapter{Introduction}
\label{c:intro}

%1. why 802.11ax, dense, DCF, contention, collision
During last decades, IEEE 802.11 achieved great success in WLAN. Enormous WiFi are deployed for its high speed and simplicity of deployment. 
The foundation of 802.11 MAC, distributed coordination function (DCF), is a random access mechanism \cite{bianchi2000performance}.
With DCF, a random access MAC, the star topology of a 802.11 WLAN result in a absolutely unfair queueing. 
Since in star topology,  access point (AP) needs to transmit all the down-link (DL) traffic, which is often more than $1/2$ traffic loading of the basic service set (BSS), while AP has only $1/n$ chance to access medium where $n$ is number of total stations including AP. It is, thus, an unfair queueing problem.
What's worse, combining effect of the unstability of random access's nature and the unfair queueing problem, once under a dense scenario, the performance will degrade severely since contention and collision will occupy the channel.
That lies the defect of legacy 802.11.
We collectively call them dense deployment problem.
The dense deployment problem not only degrades throughput but also waste much energy.

%2. ax feature, MU, central control, but OFDMA-random access
Previous amendments, 802.11n and 802.11ac called high throughput (HT) and very high throughput (VHT) respectively which are aimed at improving throughput, only mitigate the dense deployment problem. 
The bottleneck at the MAC efficiency is not resolved.
Thus, 802.11ax task group is issued targeted at high efficiency WLAN (HEW), improving quality of experience (QoE) and power save.
Confronting the dense deployment problem, 802.11ax permits AP of central control, to schedule both down-link (DL) and up-link (UL) transmission so that contention is much reduced and unfair queueing problem could be resolved.
IEEE 802.11ax also issues multi-user (MU) PHY which is supported with Orthogonal Frequency Multiple Access (OFDMA), and a special control frame called trigger frame (TF) to implement trigger-based MU UL\cite{draft_ax}. %\cite{dengquality}
Accordingly, multi-channel random access, the focus of this paper, is of course implemented in 802.11ax, named OFDMA-based random access, since random access is an efficient way for stations to transmit bandwidth request, buffer status report (BSR) etc. 
Actually, the new MAC is based on DCF since it helps co-exist among BSS and other systems. The difference is that the DCF mainly works on AP which means AP needs to access channel following DCF procedure, while HE-STA (802.11ax STA) is mostly scheduled by AP and even the random access procedure is initialized by AP. 

%3. related work, random acces history ? , clarify OFDMA-based random access is special
% SU/MU; aloha/CSMA; BEB/UB; saturated/unsaturated
Random access is a classical topic in data network which has evolved multiple variants with various physical layer.
Random access originates from Aloha and slotted Aloha in single-user channel. Then CSMA works as typical collision resolution  \cite{kleinrock1975packet}, which is accepted by IEEE 802.11 named CSMA/CA. 
The backoff mechanism, which is about retransmission when failure occurs, is also important in random access.
Random access has been a popular approach to MAC on unlicensed band for a long time, while cellular network  only implements random access for initial up-link access. 
With OFDMA, i.e., MU channel, randomness extends from time domain to frequency domain, 2-dimension. In cellular network, IEEE 802.16 and 3GPP LTE use a multi-channel slotted Aloha.
In the literature, plenty of works focus on the multi-channel random access, and most of them work on cellular network.
\cite{choi2006multichannel} designs a 1-persistent type retransmission, i.e., no exponential backoff, to achieve a fast access.  
In \cite{zhou2008efficient}, a closed-form expression of throughput for OFDMA system is firstly given.
Many works compare performance of two backoff mechanism, binary exponential backoff and uniform backoff  \cite{zhou2008efficient} \cite{seo2011design} \cite{kim2012performance}, which are implemented by IEEE 802.16 and 3GPP LTE respectively.  \cite{wei2015modeling} specifies a model estimating transient behavior of OFDMA system.
%many metrics: throughput, mean and variance of access delay, stability
All above works about OFDMA random access is an Aloha-type access in cellular network.
In addition, \cite{GeneralizedOFDMACSMACA} is one of a few works for 802.11. It generalizes CSMA/CA to OFDMA system for 802.11.

%4. our work
% the first to analyze the 802.11ax random access
Though OFDMA and multi-channel random access have been employed by IEEE 802.16 and 3GPP LTE for a long time.
It is the first time for 802.11ax to issue OFDMA and OFDMA-based random access, which is a huge evolution for 802.11.
And as far as we know, this is the first paper analyzing IEEE 802.11ax OFDMA-based random access. 
It employs binary exponential backoff and MU PHY under Trigger-based MU UL, which is different from \cite{GeneralizedOFDMACSMACA}.
Since \cite{bianchi2000performance} proposes an accurate Markov chain model for DCF, we could also reuse this model to generate another Markov chain for 802.11ax to precisely depict the OFDMA-based random access.
We assume stations are under saturated condition which means they always have packets to transmit.
The saturated analysis is based on the key assumption of independent collision probability $p$ whatever the packet is retransmitted or not.
Simulation validates our model to be accurate.
Then, we estimate the maximum system efficiency and minimum access delay. 
Since OFDMA-based random access has dynamic and more complicated parameter sets than legacy 802.11, we evaluate effect of a variety of parameter sets and at last propose rules for AP to configure the parameter set. 

The paper is organized as follows.
More explanations of 802.11ax features are given in section \ref{sec_ax_feature}. 
In section \ref{sec_RA_illu}, a detailed illustration of OFDMA-based random access procedure is presented.
Section \ref{sec_sys_model} contains the system model and performance analysis, including system efficiency and access delay, of the random access mechanism. 
Then section \ref{sec_model_val} shows simulation results compared with analysis results, which validates the model.
In section \ref{sec_max_min}, additional considerations on optimal performance are carried out. 
Section \ref{sec_perf_eval} gives performance evaluation under various parameter sets and at last propose rules for configuring the parameter set.
Conclusion remark is given in section \ref{sec_conclu}.

%\begin{figure}
%\centering
%\includegraphics[width=0.45\textwidth]{kl}
%\caption{kl-distance}
%\label{kl}
%\end{figure}
%
%\begin{table}[t]
%\begin{center}
%\begin{tabular}{lcc}
%
%\hline
%                    &  {\small Itti's method}     & {\small Fuzzy growing}    \\
%\hline
%{\small Precision}           &  0.4475    & 0.4506 \\
%{\small Recall}              &  0.5515    & 0.5542 \\
%\hline
%
%\end{tabular}
%\caption[Evaluation of FOA sets]{\small Evaluation of FOA sets. } \label{t:FOA}
%\end{center}
%\end{table}
